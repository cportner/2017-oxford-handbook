\documentclass[letterpaper,12pt]{article}

\usepackage{mathpazo}
\usepackage[longnamesfirst]{natbib}
\usepackage[flushleft]{threeparttable} 
\usepackage{booktabs}
\usepackage{rotating} 
\usepackage{amssymb} 
\usepackage{amsmath}
\usepackage{caption} 
\usepackage{dcolumn} 
\usepackage{setspace}
\usepackage[longnamesfirst]{natbib}
\usepackage[font=scriptsize]{subfig}
\usepackage[pdftex,colorlinks=true,linkcolor=black,citecolor=black]{hyperref} 
\usepackage[margin=1.0in]{geometry} 
\usepackage{multirow}

\newcommand{\mco}[1]{\multicolumn{1}{c}{#1}}
\newcommand{\mct}[1]{\multicolumn{2}{c}{#1}}


%opening
\title{Fertility Issues in Developing Countries} 

\author{Claus C P\"ortner\\
    Department of Economics\\
    Albers School of Business and Economics\\
    Seattle University, P.O. Box 222000\\
    Seattle, WA 98122\\
    \href{mailto:cportner@seattleu.edu}{\texttt{cportner@seattleu.edu}}\\
    \href{http://www.clausportner.com}{\texttt{www.clausportner.com}}\\
    \& \\
    Center for Studies in Demography and Ecology \\
    University of Washington\\ \vspace{2cm}
    }

\date{February 2017}

\doublespacing

\begin{document}
\graphicspath{{../figures/}}
\DeclareGraphicsExtensions{.jpg,.jpeg,.pdf,.mps,.png}

\maketitle
\thispagestyle{empty}


\newpage

\section{Introduction}

Despite a common perception that fertility is very high in developing countries,
the truth is substantially more complicated.
Figure \ref{fig:tfr} shows that there has been an astonishing decline in total 
fertility rate (TFR) in developing countries over the last half century.%
\footnote{
TFR is the number of children a women entering her reproductive life
would have if she had children following the age-specific fertility
rates observed at that point in time.
Hence, it is composite or snapshot measure of current fertility
behavior.
}





\section{Conclusion}


\bibliographystyle{aer}
\bibliography{collection}


\end{document}
