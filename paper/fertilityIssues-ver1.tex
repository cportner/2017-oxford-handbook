\documentclass[letterpaper,12pt]{article}

\usepackage{mathpazo}
\usepackage[longnamesfirst]{natbib}
\usepackage[flushleft]{threeparttable} 
\usepackage{booktabs}
\usepackage{rotating} 
\usepackage{amssymb} 
\usepackage{amsmath}
\usepackage{caption} 
\usepackage{dcolumn} 
\usepackage{setspace}
\usepackage[longnamesfirst]{natbib}
\usepackage[font=scriptsize]{subfig}
\usepackage[pdftex,colorlinks=true,linkcolor=black,citecolor=black]{hyperref} 
\usepackage[margin=1.0in]{geometry} 
\usepackage{multirow}

\newcommand{\mco}[1]{\multicolumn{1}{c}{#1}}
\newcommand{\mct}[1]{\multicolumn{2}{c}{#1}}


%opening
\title{Fertility Issues and Policy in Developing Countries} 

\author{Claus C P\"ortner\\
    Department of Economics\\
    Albers School of Business and Economics\\
    Seattle University, P.O. Box 222000\\
    Seattle, WA 98122\\
    \href{mailto:cportner@seattleu.edu}{\texttt{cportner@seattleu.edu}}\\
    \href{http://www.clausportner.com}{\texttt{www.clausportner.com}}\\
    \& \\
    Center for Studies in Demography and Ecology \\
    University of Washington\\ \vspace{2cm}
    }

\date{May 2017}



\begin{document}

\maketitle
\thispagestyle{empty}

\begin{abstract}
[need an abstract of up to 130 words]
\end{abstract}

\newpage 

\doublespacing

\section{Introduction}

Despite a common perception that fertility is very high in developing
countries, the truth is substantially more complicated.
Figure \ref{fig:TFR} shows that there has been an astonishing decline in
most developing countries' total fertility rate (TFR) over the last half
century.%
\footnote{
TFR is the number of children a women entering her reproductive life
would have if she her childbearing following the age-specific fertility
rates observed at that point in time.
Hence, it is composite or snapshot measure of current fertility
behavior rather than any woman's actual fertility behavior.}
Half a decade ago, the TFR was around 7 children throughout the world,
with the exception of Europe and Central Asia.
The most recent data show, however, that, with the exception of
Sub-Saharan African, the TFR is now either below or only slightly above
the replacement level of 2.1.
Despite this rapid decline in fertility, total population is still
growing in many of these regions because there are still many more young
people than older people and these young people either have not entered
reproductive age or are just starting out.

\begin{figure}[hp]
    \centering
    \caption{Total Fertility Rates by Region from 1967 to 2015 for Developing Countries}
    \includegraphics[width=0.75\linewidth]{../figures/totalFertilityRates.pdf}
    \label{fig:TFR}
\end{figure}

If fertility levels in developing countries are quickly approaching 
those of developed countries and there is rapid urbanization and increasing labor
force participation among women, does this Handbook even need a chapter
focused on fertility in developing countries? 
The goal of this chapter is to highlight areas in which a separate focus 
on developing countries is still relevant, what the recent developments 
in research has been, and most importantly, what I consider to be the 
main outstanding issues.
I begin with the question of why the changes in Sub-Saharan Africa's 
fertility behavior appear to be different from the other developing countries. 
I then cover three areas where there are still likely to be substantial
differences between developed and developing countries.
First, how couples time their fertility, and especially the interplay of
the timing of the first birth and educational attainment.
Second, how differences in status and power in the household---what economists 
refer to as bargaining power---affect fertility decision.
Third, the role in fertility decisions of the the strong preference for boys 
over girls that still exist in many developing countries. 
The final part of the chapter reviews what we know about the effects
of different types of population policies in developing countries.


\section{Sub-Saharan Africa}

Both in terms of the degree of decline and the overall level, the
outlier in Figure \ref{fig:TFR} is Sub-Saharan Africa.
Sub-Saharan Africa now has an average TFR that is about twice as large
as the other regions.
Most of the projected future increase in world population is therefore
likely to come from Sub-Saharan Africa \citep{Gerland2014}.%
\footnote{
Currently Africa is home to about 1 billion people, but this will
increase to between 3.1 and 5.7 billion by the end of the century at
current projected fertility rates.}
The most important issue from a policy standpoint is why the fertility
decline in Sub-Saharan Africa has moved at a much slower pace than the
other regions and even appears to have stalled in some countries
\citep{Ainsworth1996a,Singh2017}.
The purpose of this section is not to provide the final answer, but
instead to highlight both how we can think about fertility decisions and
suggest possible answers.
A caveat to the following discussion is that there are important
differences across countries within Sub-Saharan Africa that we
cannot cover in sufficient detail.

Broadly speaking there are two competing approaches to explaining
fertility decisions.%
\footnote{
This is clearly a simplification but it serves to illustrate the
differences in approaches.}
One sees fertility preferences as the main driver of fertility and
considers preferences malleable and mainly determined by cultural
factors and transmission of ideas of ideal family size across groups and
over time.
Under this approach the main constraints on reaching desired fertility
is the level of access to family planning and contraceptives.

The other approach sees the decision on fertility as driven by the
trade-off between the cost of children and the return to children, which
can either be monetary or the utility of having offspring.
In this approach parents are assumed to be able to control fertility
even in the absence of modern contraceptives.
Hence, although the lower cost of preventing births---for example easier
access to modern contreceptive---will still lower fertility in this
approach, the resulting decline in fertility is assumed to be much
smaller than in the first theory.

\begin{figure}[hp!]
    \centering
    \caption{Under-Five Mortality Rates by Region from 1967 to 2015 for Developing Countries}
    \includegraphics[width=0.75\linewidth]{../figures/childMortalityRates.pdf}
    \label{fig:mortality}
\end{figure}

Both theories consider the surviving number of children as the main
outcome of interest.
One possible explanation for the slow decline in fertility could
therefore be that mortality in Sub-Saharan Africa is higher than in the
other regions.
Figure \ref{fig:mortality} shows the pattern in under-five mortality rates 
across the same regions as above over time.%
\footnote{
The under-five mortality rate is the probability of dying between birth and 
exactly five years of age, expressed per 1,000 live births.
}

The improvements in mortality risk over time are truly astonishing.
Over the last half-decade under-five mortality in developing countries has
fallen from close to 175 to below 50 per 1,000 live births.
Sub-Saharan Africa, however, lags substantially behind other regions.
Despite a massive improvement from a situation where more than a quarter
of all children born did not live to see their fifth birthday to about
80 deaths per 1,000 births, the current mortality rate is still more
than three times larger than that of the other regions (with the
exception of the Middle East and North Africa).
Although mortality is likely part of the explanation, it cannot be the
full explanation.
One way to see this is to compare fertility levels across regions when
they reach the same mortality level.
For example, mortality in Sub-Saharan Africa is currently at the same level 
as mortality was in South Asia around the turn of the century, but fertility 
is about 1.5 children higher in Sub-Saharan Africa now than it was in South 
Asia at the turn of the century.

If higher child mortality is not the explanation, what might lead to the
higher fertility in Sub-Saharan Africa? 
Demographers, following the first approach described above, have argued 
that the two main reasons for the slow decline in fertility in Sub-Saharan 
Africa are the high ideal family size that is still in place and a 
substantial ``unmet need'' for contraception 
\citep{Bongaarts2013a,Casterline2017,Singh2017}.
Contraceptive use is, indeed, lower in Sub-Saharan Africa than the other
regions, but countries have managed to reduced fertility even in the
absence of access to modern contraceptives
\citet{Schultz1985,Galloway1987,Bailey1998,bengtsson06}.
Furthermore, fertility in Sub-Saharan Africa has been---and still 
is---characterized by longer birth intervals than in other regions even
in the absence of access to modern contraception 
\citep{Caldwell1992,Moultrie2012,Casterline2016}.
The longer birth intervals arise predominantly from longer postpartum 
sexual abstinence and extended periods of breastfeeding compared to 
other regions.
To the extent that these behaviors are the result of conscious decisions, 
the longer spacing suggests that people are able to control fertility.%
\footnote{
It is still possible that fertility is higher than desired because the
higher cost of preventing ``accidental'' conceptions.
This would explain why the estimated effect of access to family planning
in Ethiopia shows a reduction in fertility of about one birth, which is
equivalent to an approximate 20\% reduction in fertility
\citep{Portner2014a}.}

Three alternative explanations may explain the slow decline: the
relative abundance of land compared to other regions; low levels of
education or at least low levels of quality education; and the role of
urbanization across regions.
Each of these is discussed in what follows.

There are two important characteristics of land access in 
Sub-Saharan Africa that may explain the higher fertility.
First, there is, on average, more land per capita in Sub-Saharan Africa 
than in the other regions.
At the median projected population growth for Sub-Saharan Africa---which
is 4.2 billion people by 2100---the population density will only be
roughly equal to that of China today \citep[p 235]{Gerland2014}.
The low density means that there is little pressure to restrain
fertility for fear of running out of land.
In fact, it is likely that there is a higher return to children in
Sub-Saharan Africa than in the other regions---or, at least, a
substantially lower cost---because the return to children working on the
family farm is higher \citep{Caldwell1992,Bongaarts2013a}.
Similarly, there are substantially higher return to having wives work on
agricultural land \citep{jacoby95,Matz2016}.
The associated polygyny in some countries also appeared to have resulted 
in a situation where the cost of the children where born by the individual 
wives, but the decision on fertility was made by the husband.
[TK check that is discussed] I return to this point below.

A second characteristics of land in Sub-Saharan Africa that also can
lead to higher fertility is that---despite its abundance---access to
land rights are in many countries controlled at the local level by 
chiefs and other local institutions rather than through market-based 
buying and selling of land.
This is important because the main method used to maintain the 
productivity of agricultural land is often through fallowing, and with less
secure land rights farmers may fallow their land for shorter periods
than those with more secure rights \citep{Goldstein2008}.%
\footnote{
See also \citet{besley95c}, who discuss other investments in land that
can secure property rights.}
The reason unsecure land rights can lead to higher birthrates is that
land is often allocated based on the number of household members.
Hence, more children, everything else equal, increases a household's 
claim to land access.
The irony here is, of course, that if everybody else follows the same
strategy, the result will be much higher fertility and little change in
the allocation of land.

For both of these potential effects of land access on fertility we,
however, have little direct information on their effects, and the role
of land in fertility is one area that calls out for future research.
This is, however, made more difficult by the need to measure land access.
Although we obviously have information on total land area for countries,
measuring arable land is often more difficult, complicating efforts to
compare land access and fertility across countries.
At the individual level we would need information on and exogenous
variation in \emph{potential} land access---that is, how much land 
parents expect to have access to for a given number of children---and 
as the second argument show land allocation is not a random process.
Specifically, if there are unobserved characteristics that influence
both how much land parents have access to and their number of children,
the relationship between current land access and number of children 
will provide a biased estimate of the effect of land access.
One approach would be to use \emph{de jure} or \emph{de facto} changes 
in land laws or programs aimed at securing land rights for farmers 
and examine their effects on fertility.

My second potential explanation for a major factor impacting fertility in
Sub-Saharan Africa is education.
The standard economic model of fertility considers the opportunity cost
of women's time to be the main factor affecting the number of children
\citep{becker91}.
As women gain more education, the cost of their time, and therefore of
childbearing and childrearing, increases, reducing fertility and leading
to better health outcomes for both women and children.%
\footnote{
It is, however, not completely clear why there is such a strong
association between education and health
\citep{Thomas1991,Glewwe1999,Kovsted2002}.} 
The better health outcomes lead to lower child mortality, which, in turn, 
further decreases fertility, because fewer births are required to reach 
a desired number of surviving children \citep{Ainsworth1996}.
The effect of education on fertility is essentially universal, making it
the main recommended way to decrease fertility \citep{schultz02}.

Fertility, however, begins to decline at higher levels of education in
Sub-Saharan Africa than in other regions and the relationship between
fertility and education may even be positive for low levels of education
\citep{Ainsworth1996,Benefo1996,Thomas1996}.
Part of the problem may be the quality of education in Sub-Saharan
Africa.
In other words, the stated number of years of education may be a worse
predictor of actual human capital accumulation in Sub-Saharan Africa
than other regions.%
\footnote{
As shown by \citet{Oye2016} this is not the same as saying that low
quality education has no impact on fertility; only that the effect
of education on fertility is lower the lower the quality of the
education received.
}

A good example of this problem is Tanzania
\citep{Galabawa2001,Wedgwood2005}.
Taken at face value, Tanzania has a very high reported education level.
This is most likely the result of the 1974 Universal Primary Education
Movement, which increased accessibility of primary education and
enrollment rates.
The problem is that the quality of education reportedly was very low.
In addition, the economic crisis Tanzania experienced in the 1980s
further lowered the quality and enrollments declined significantly.
Hence, it is unclear to what extent reported education levels reflect
women's actual human capital.
The result is that education does not appear to have as substantial an
effect on fertility in Tanzania as other found elsewhere
\citep{Alam2016}.



The final potential explanation for differences in TFRs across regions 
is the role of urbanization.
One topic that seems to be essentially absent in discussions about
fertility and its determinants in Sub-Saharan Africa is the difference
between urban and rural areas.
As a rule, all regions have had and still have higher fertility in rural
areas than in urban areas.
This is directly in line with what we expect.
The cost of children is clearly higher in urban areas than in rural
areas, even for women with the same amount of education---and therefore
the same opportunity cost of time.
Sub-Saharan Africa is no different.
An example is Ethiopia in 2011, where the overall TRF is 4.8, but that
masks a TFR of 5.5 in rural areas and only 2.6 in urban areas
\citep{Central-Statistical-Agency/Ethiopia2012}.
Part of the explanation for the lower fertility is the higher average
education level of women in urban areas than in rural areas.
But, even for women with the same education level, fertility is lower in
urban areas than in rural areas \citep{Ainsworth1996}.

There has, however, not been a systematic examination of how fertility
varies with education in urban areas across different regions.
If predicted fertility is similar across regions for the same level of
education, that would suggest that Sub-Saharan Africa is not inherently
different.
A lower ``return'' to education could either be an indication that the
quality of education is lower, that the opportunity cost increases with
higher education is not as high in Sub-Saharan Africa as in other areas
(either because of the lower quality or because of lower levels of
development), or it could suggest that there is something inherently
different in what determines fertility in Sub-Saharan Africa than in
other regions.

\section{Timing of Fertility}

How couples decide on when to have their children is relevant both 
because it provides us with an idea of how good people are at controlling 
their fertility and because timing of births may impact the health of both 
a mother and her children.%
\footnote{
The idea that people plan their fertility is not without opposition.
\citet{Timaeus2008} and \citet{Moultrie2012}, for example, argue that
``postponement'' of births---that is, simply not having a birth now---is 
different from decisions on the spacing between births and the overall
number desired.
}
We know, however, surprisingly little about what determines the timing
of births in developing countries.
Especially with more and more women entering the labor force in
developing countries, understanding how timing decisions are made will
be important for the design of suitable policies.
The lack of research is partly because of data limitations and partly
because of the difficulty in identifying the causal relationship between
timing and other decisions, such as labor supply.
The two sub-areas where we do have some information are the timing of
first births and how the sex of the last child affects timing of the next birth.
This section covers the timing of first birth and leaves the other
for the sections below on sex preference.

Having a first birth earlier in life is associated with lower educational 
attainment, higher completed fertility, and worse health and labor outcomes.%
\footnote{
See also the discussion of the literature on the association between short 
birth spacing and child health more generally in \citet{Casterline2016}.
}
This is, however, not necessarily indicative of a causal relationship
between earlier first birth and the other outcomes.
A woman who, for example, has a lower expected return to education may
decide that using contraceptives is not worth the cost and therefore
would be more likely to conceive and subsequently drop out of school.
Furthermore, as long as fertility is well below natural fertility
levels, having an earlier birth will not, in itself, increase total
fertility.%
\footnote{
Natural fertility is the level of fertility that would prevail in a
population that makes no conscious effort to limit, regulate, or control
fertility.}

For this reason, most of the literature has focused mainly on what
determines the timing of first births and, to some extent, on whether
women are more likely to drop out of school after their first birth.
In the relatively small literature on timing of first births, there are
two main approaches to trying to identify a causal relationship between
timing of first birth and other outcomes.
One is to look for variables that plausibly only affect one or the
other, with no direct effect on the other outcomes, and then jointly
estimate the various decisions.%
\footnote{
This approach is often combined with restrictions on the correlation of
error terms across decisions.
} 
The other approach is experimental, where researchers randomly grant 
access to a program that is believe to influence one of these decisions 
and then examine whether the timing of births and the other outcomes 
are affected by the program.
Independent of method, the results suggest that increasing education is
important in delaying marriage and first birth
\citep{Duflo2015,Marchetta2016}. 

The downside of both approaches is that we cannot learn much about what
completed fertility is going to look like.
Even experiments that follow people for an extended period, like the
seven years in \citet{Duflo2015}, only extends to the beginning of the
prime childbearing years, ages 20 to 30.
An important caveat is also that the effects of interventions may disappear 
quickly after the end of the program \citep{Baird2016}.



\section{Bargaining Power}

An important question is what happens when husband and wife do not agree
on the desired number of children.
The original literature on this question mainly showed that families do
not necessarily behave as if husband and wife have similar preferences.
One way to see this is to look at unearned income---mainly income that 
does not come from the application of skills and labor---that can be
attributed to either the wife or the husband.
If the number of children born changes with shifts in the distribution
of this income, this indicates that the partners have different
preferences.

Results from Thailand as of the 1980s show that women with more
``bargaining power'' spent less time working and preferred to have more
children \citep{Schultz1990}.
The result that women prefer more children is not generally supported,
however, and in most cases men have a higher preferred number of
children than women \citep{Westoff2010}.
An example of this is Malaysia where both ethnic Chinese and ethnic
Malay husbands had a higher ideal number of children than their wives,
although the ideal numbers were below the actual number of children for
both partners \citep{Rasul2008}.

Sub-Saharan Africa is often considered a special case when it comes to
different preferences for the ideal number of children across husband
and wife.
The father bears less of the cost of children than in other developing
countries because of the family structure, especially in West Africa
\citep{Caldwell1992}.
It would seem that in cases like this, it would be beneficial to provide
women with more control over use of contraceptives.
An experiment in Zambia tried exactly that \citep{Ashraf2014}.
One group of women was given an individual voucher for free and
immediate access to contraceptives.
Because the most popular contraceptives are injectables, they were able
to hide contraceptive use from their partners if they wanted to.
In the other group, husbands were given the vouchers and both the
husband's and wife's signatures were required to redeem it.
As expected, the women who needed their husband's signature were less
likely to visit a family planning nurse and less likely to use
injectable contraceptives.
As a result, these women were also more likely to have a birth.
The caveat is that women who could potentially conceal their use of
contraceptives reported significant reductions in happiness, health, and
ease of mind, compared to the women in the group where both signatures
were required.

When discussing differences in preferred number of children across men
and women it is important to realize that in some countries men do end 
up with more children than women \citep{Field2016}.
Using data from eight Sub-Saharan African countries, men have, on
average, more children than women of the same cohort in seven out of the
eight countries.
The gaps are large, ranging from 0.8 children in Zambia to 4.6 children
in Burkina Faso, but appear to be decreasing over time.
This pattern is consistent with men partnering with younger women in a
situation where the population is growing and also with polygyny.
This indicates that that differences in desired number of children are
often mirrored in differences in actual achieved fertility.
The implication is that there might not be an innate contradiction
surrounding fertility behavior within couples, at least in
countries with growing populations \citep{Field2016}.

\section{Sex Preference}

An especially important aspect of intrahousehold allocation is the
preference for children of a specific sex.
The dominant version is a strong preference for sons in many countries,
most notably in India and China.
The literature is somewhat fuzzy on what exactly constitutes son
preference.
One popular version is that for their ideal number of children, parents
would prefer to have more sons than daughters, and the strength of son
preference is then measured by how many more sons than daughters a
family wants.%
\footnote{
In India's National Family Health Surveys conducted, in and Demographic 
and Health Surveys in general, son preference is measured as part 
of the general fertility questions.
The first question is how many children a woman would have if she could 
choose exactly how many.
This if followed by the sex preference question:
``How many of these children would you like to be boys, how many would you 
like to be girls and for how many would the sex not matter?''
}
This measure of son preference is commonly used in the literature
\citep[e.g.]{clark00,Jensen2009,Hu2015}.
Other versions are possible.
Parents might, for example, have a preference for one son, but once that
one son is secured they do not have strong preferences for the
distribution between sons and daughters for the remaining children.

Before prenatal sex determination became available, most research
focused on the impact of son preference on fertility decisions and
spacing between births.
This literature showed clearly that in areas with son-preference,
families were more likely to stop childbearing after the birth of a son
than after the birth of a daughter \citep[see, for
example,][]{Das1987,Arnold1997,clark00,filmer09}.
Furthermore, in the absence of sex-selective abortions, son-preference
often leads to a shorter birth interval if the previous birth was a
daughter \citep[see, for
example,][]{Das1987,Rahman1993,Pong1994,Haughton1996,Arnold1997}.
The resulting shorter spacing is thought to be associated with worse
health outcomes for the girls
\citep{arnold98,Whitworth2002,Rutstein2005,Conde-Agudelo2006}.
There is also evidence that girls are underreported in China as a result
of strong son preference combined with the one-child policy
\citep{Merli2000}.

With the introduction of amniocentesis, ultrasound, and chorionic
villus sampling (CVS), it became possible to tell the sex of a fetus 
and abort the pregnancy if the fetus was not of the preferred sex.
By itself, this may not have led to substantial changes, but combined
with lower desired fertility as in India or forced lower fertility as in
China, the availability of prenatal sex determination had substantial
impacts on the sex ratio.
It is easy to see how declining fertility can increase the use of sex
selection.
Consider a family that wants one son.
If the family is willing to have up to 4 children, the probability of
having a son is more than 94 percent, even without sex selection, and
that increases to almost 99 percent if the family is willing to have up
to 6 children.%
\footnote{
The probabilities of not having a son are 48.8 percent for one child,
23.8 percent for two children, 11.6 percent for 3 children, 5.7 percent
for 4 children, 2.8 percent for 5 children, and 1.4 percent for 6
children.}
If the desire is instead for one son \emph{and} a maximum of two
children, there is a 24 percent chance that the family will have to
resort to sex selection to achieve both targets.

There is relatively little empirical analysis of the effects of
fertility on sex selection using individual-level data
\citep{park95,Ebenstein2011}.
At the country level \citet{Bongaarts2013} shows how sex ratios at birth
are only elevated for countries with lower fertility and
\citet{Bongaarts2015} use national-level estimates of the relationship
between the sex ratio at birth and fertility as part of their prediction
of the number of missing women past and present.
Furthermore, simulations suggest that in Korea the introduction of sex
selection changed family size little, but did result in abortions of
female fetuses equal to about 5 percent of actual female births
\citep{park95}.
For China allowing a three-child policy has been predicted to increase
the fertility rate by 35 percent, but also reduce the number of girls
aborted by 56 percent \citep{Ebenstein2011}.%
\footnote{
Strong son preference does not, however, automatically lead to high use
of sex selection.
One example of this is Turkey \citep{Altindag2016}}

Recent research suggests that son preference in India, when measured as
ideally having more boys than girls, is decreasing over time and with
higher education \citep{bhat03,pande07}.
This may, however, be an artifact of the retrospective nature of the way
desired fertility questions are often asked.
If we instead ask parents what the desired sex composition would be for
a specific number of children, the results are different 
\citep{Jayachandran2017}.%
\footnote{
Parents were asked about their desired sex composition for their
children's children to avoid the retrospective nature of the 
standard sex preference questions.
The question was:
``Suppose your son/daughter [the specific grade 6 or 7 child we surveyed]
was going to have $N$ children. 
How many of them would you want to be boys
and how many would you want to be girls?''
The $N$ was a randomly chosen number between one and five.
}
The lower the given number of children, the higher the proportion of
boys to girls was.
This result is consistent with evidence that women with more education
and urban women are more likely to use sex selection
\citep{Portner2015b}.
These women have higher costs of children and therefore lower desired
fertility.

\section{Policies}

Even though most people automatically think of family planning programs
when population policy in developing countries is mentioned, any policy
that changes the opportunity cost of time or affects the distribution of
bargaining power within the household will also affect fertility.
I therefore cover both standard family planning programs and other
policies that impact fertility.

Despite a substantial and long-standing interest in the effectiveness of
family planning programs, there is relatively little convincing
empirical evidence.%
\footnote{
For a more in-depth discussion of both the history of family planning
programs and the literature, see \citet{Miller2016}.
An older review of the literature, focusing on whether access to family
planning changes preferences for number of children, is in
\citet{Freedman1997}.
\citet{Singh2012} provide recent estimates of the use and need for
contraceptives in the developing world, together with cost of providing
contraceptive services.}
The lack of evidence is mainly the result of the challenges in measuring
family planning program's impacts.
First, studies of family planning programs have often covered periods of
rapid economic development and fertility decline, making it difficult to
isolate the effects of family planning programs from the changes in the
economy.
Second, existing studies have largely ignored heterogeneous impacts,
especially whether women with different education levels respond
differently to family planning.
Evidence from the US shows that better-educated women and less-educated
women are equally efficient users of modern contraceptives, but
better-educated women are more efficient at using ``ineffective''
contraceptive methods such as withdrawal or rhythm
\citep{Rosenzweig1989}.%
\footnote{
This is supported by the differences in the effect of the roll-back
of abortion access in Romania, which resulted in bigger increases
in fertility for less-educated women than better-educated women 
\citep{Pop-Eleches2010}.
}
This suggests that the effect of family planning should be stronger, the
lower the education levels, but few studies address this for 
developing countries.


Finally, rigorous study is hampered by the challenge of non-random
program placement \citep{rosenzweig86,pitt93,Miller2016}.
Part of the problem is that unobserved characteristics of both women and
the areas they live in might lead some women to be more likely to both
have access to family planning and to use it.
In that case, simply looking at the correlation between use of family
planning and fertility will overstate the strength of the relationship
in the general population.

Randomizing the allocation of programs and comparing the outcomes of
interest between treatment and control areas could overcome the
non-random program placement problem.
Although theoretically superior, such experiments have several drawbacks
in practice.
First, there are concerns about the external validity of experiments,
which are often small in scale.
Additionally, non-compliance of randomization can further decrease the
power of the experiment \citep{Desai2011}.
This is especially a problem for programs like family planning where the
randomization takes places at the community level rather than at the
individual level.
Second, because of the cumulative nature of fertility, an experiment
must run for a substantial period before one can assess the effect on
fertility.
This is, for example, a likely explanation for the absence of an
increase in contraceptive use from an experiment in Ethiopia
\citep{Desai2011}.
Even if an effect is found, these short-run effects may simply reflect
changes in spacing-patterns rather than changes in the overall number of
children.
When run for too short a period, experiments may also be prone to
short-term health scares, such as the one experienced by an experiment
in Zambia where people were led to believe that injectable contraceptives
contained HIV resulting in a four months national ban \citep{Ashraf2009}.%
\footnote{
The published version of this paper does not mention the scare
\citep{Ashraf2014}.}

The Matlab family planning program from Bangladesh is the least likely
to suffer from these drawbacks.
It began in 1978, when the International Centre for Diarrhoeal Disease 
Research, Bangladesh (icddr,b) introduced a family planning program
in 70 of the 149 villages covered by the demographic surveillance system
in the area.
The icddr,b family planning program was characterized by an outreach
program, consisting of home visits by trained female outreach workers.
By 1984, fertility was 24 percent lower in the villages that received
the intensive family planning program compared to the villages that
received only the standard family planning program \citep{Phillips1988}.
More recent work using the same villages with data until 1996 finds a
decline in fertility of about 15 percent in the program villages
compared with the control villages, despite rapid declines in fertility
in the control villages \citep{Sinha2005,Joshi2007}.
These results reflect, however, a level of program intervention and
intensity that some argue are unlikely to be sustainable
\citep{pritchett94a}.%
\footnote{
Per woman reached, the program cost 35 times more than the standard
government family planning program and each averted birth cost \$180 in
1987, 1.2 times GDP per capita at the time.}
Using a quasi-experimental approach, the Navrongo Project in northern
Ghana also found an initial 15 percent reduction, although that was 
based on only the initial 3 years of the program \citep{Debpuur2002}.
After 15 years there was no evidence of the project affected long-run
fertility outcomes \citep{Phillips2012}.

If longitudinal data were collected in parallel with the introduction of
the program, program effects can be estimated using fixed effects,
provided there are enough areas that receive a program between the
(minimum) two survey rounds and provided the period between the rounds
is long enough.
Examples from Indonesia of this approach found a negative (but not
statistically significant) effect on fertility, responsible for only 4
to 8 percent of the decline in fertility from 1982 to 1987
\citep{pitt93,Gertler1994}.
Longitudinal data are, however, most often not available or cover too
short periods, in practice limiting researchers to using cross-sectional
data.%
\footnote{
There are also additional problems with using fixed effects, such as
measurement error bias.
For a discussion of this and other problems in the study of family
planning see, for example, \cite{angeles98}.}

If neither experiments nor longitudinal data are available, one approach
is to use variables that influence program placement but are unrelated
to individual fertility, what is known as the instrumental variable (IV)
approach.
This is the least appealing approach when trying to identify the causal
impact of family planning because it relies heavily on the choice of
variables that affect program placement without any direct test for
whether these variables are appropriate.
Despite these drawbacks, it is often the best that can be done given the
constraints.

Using this approach, a woman in Tanzania exposed to family planning
throughout her fertile lifespan is found to have 4.13 children compared
with 4.71 children in the absence of family planning programs
\citep{angeles98}.%
\footnote{
See also \citet{Angeles2005} on Indonesia and \citep{Angeles2005a} on
Peru.}
Lingering concerns remain, however, that some of the variables used to
identify placement (such as child mortality levels and the presence of
other family planning services) may also be correlated with unobservable
variables that influence both placement and fertility decisions.
Examining the difference in effects of providing subsidies for
contraceptives or expanding access to previously not served areas,
results from Indonesia show that subsidizing contraceptive with
about half of the total cost lowers fertility by about 3 to 6 percent, 
whereas expanding the distribution network by
one standard deviation lowers fertility by about 12 percent
\citep{Molyneaux2000}.
These results are consistent with what is found for Profamilia,
Columbia's family planning program, which reduced lifetime fertility by
around half a child, equivalent to less than 10 percent of the sharp
decline in fertility over the period the program was implemented
\citep{Miller2010}.

While most studies find an effect of about half a child,
\citet{Portner2011} find a substantially higher effect of access to
family planning in Ethiopia.%
\footnote{
The half a child reduction is also found in Romania using that country's
ban on abortion and other birth control, with bigger effects the less
educated the woman \citep{Pop-Eleches2010}.}
Access to family planning reduces completed fertility by more than 1
child among women without education, which is equivalent to a 20-25
percent reduction.
No effect is found among women with some formal schooling, suggesting
that family planning and formal education act as substitutes, at least
in this low income, low growth setting.%
\footnote{
These results run counter to the argument in \citet{Feyisetan1996} that
low education is a constraining factor in the uptake of contraception,
although their data cover a period before long-acting injectable
contraceptives became widely available.
There is mixed evidence from the Matlab family planning program on
whether program acted as a substitute for female education in the
reduction of fertility \citep{Sinha2005,Joshi2007}.}
Both highlight the importance of examining how access to family planning
can vary depending on the recipients' characteristics.

A very different approach to understanding how family planning access
affects fertility is to examine the response to disruptions in access or
substantial changes in the price of contraception.
These are---by their very nature---often temporary and therefore cannot
tell us much about final fertility outcomes, but they do have the
advantage here of mostly being exogenous to the individual women.
That is, the disruption in supply of contraceptives comes as a surprise
and is independent of the individual women's initial demand for
contraception.

The 1997 financial crisis in Indonesia led to very large changes in
prices of contraceptives, because it reduced the government's ability to
subsidize the price of contraceptives \citep{McKelvey2012}.
Despite the large price changes there were few changes in either the
choice of method or the decision to use contraceptives.
This result holds even for the poorest couples who are most likely to
rely on the subsidy for access to contraceptives.

The United States' implementation of the Mexico City
Policy\footnote{Also often referred to as the ``Global Gag Rule''.
It was originally implemented in 1984 under the Reagan administration,
rescinded during Clinton, then reinstated again under Bush, and
rescinded under Obama.}, which forbid funding non-governmental
organizations (NGOs) that perform or promote abortion services, has also
been used to identify the effects of access to contraceptives because
most of the NGOs affected also provide subsidized contraceptives.
In Ghana, contraception availability and use were reduced during the
periods the policy was in effect \citep{Jones2015}.
This led to significant increases in conception for rural women, but not
for urban women.
Within rural areas, the poorest women were the most affected with a 7 to
10 percent higher fertility, whereas less poor women saw increases of 3
to 6 percent.
Perversely for a policy aimed at reducing abortions the effect was
exactly the opposite.
Rural Ghanaian women in the upper three wealth quintiles aborted 4 out
of every 10 additional pregnancies that were the result of the lower
contraception availability.
The poorest women did not change their abortion behavior and therefore
ended up with significantly more children.

That the policy increases the use of abortions is supported by analyses
of cross-country data for Sub-Saharan Africa \citep{Bendavid2011}.
Using data from 1994 to 2008, countries were divided into ``high
exposure'' and ``low exposure'' countries, depending on the level of
financial assistance per capita provided by the United States when the
policy was not active.
The probability of having an abortion for a woman in a ``high exposure''
country was more than twice that of a woman in a ``low exposure''
country when the policy was in effect.
Furthermore, there was no apparent difference in abortion rates when the
policy was not in effect and the abortion rate in ``high exposure''
countries began to rise only after the policy was reinstated in 2001.
Finally, the use of modern contraceptive stopped increasing after 2001
in ``high exposure'' countries, whereas ``low exposure'' countries
continued to see increases in contraception use.

A different type of supply interruption is found in the Philippines,
where a scheduled phase-out of international donations of contraceptives
combined with decentralization of the responsibility of providing
contraceptives and supply chain issues lead to substantial variation in
the availability of contraceptives over time and across area
\citep{Salas2014}.
Both supply reductions and swings in the supply of contraceptives lead
to significant increases in the number of births.
The poorest women, those living in rural areas, and those with less than
a high school education, were the most affected by supply fluctuations.
The Philippines were also the location of an outright ban on modern
contraception in the city of Manila.
Comparing Manila and other cities in the capital region, and assuming
that these cities would have had similar fertility trends in the absence
of the ban, the ban resulted in an approximately 3 percent increase in
the number of children \citep{Dumas2017}.
The effect is relatively larger, the younger the mother.

Probably the most notorious approach to population control is China's
one-child policy, which began in 1979.
Households that exceeded their ``birth quota'' were penalized, but the
birth quota depended on ethnicity and later on the sex of the first-born
child \citep{Li2005}.%
\footnote{
Ethic minority women were allowed two children until the late 1980s.}
Furthermore, there was substantial heterogeneity in how the policy was
implemented across regions.
Women in urban areas who exceed their birth quota were, for example,
generally punished much more severely than women in rural areas.
Despite the scale of the program, there has been little research
directly on its effects on fertility.
One study used differences in implementation across areas and found an
11 percent reduction in the probability of a second birth; the policy
was more effective among urban and well-educated women \citep{Li2005}.
Interestingly, the policy had almost no effect on the least well-off
group, which consists of rural residents with little or no education.
To the extent that having fewer children translates into better health
and education outcomes for children as suggested by \citet{becker73},
this disparity in the effect may lead to increased inequality.%
\footnote{
Although, \citet{Rosenzweig2009} find only small increases in education
investments as a result of the policy and its associated decline in
fertility.}

Whether or not family planning programs have a substantial effect on
fertility, it is possible that they can improve the well-being of both
women and children simply through providing better control over timing
of births.
There is, however, even less solid research on the long-run effects on
other outcomes than there is for the effect on fertility.
Part of the problem is that identifying the causal effect of programs is
even harder for other outcomes than fertility, because a woman's
fertility and her investment in her children are likely driven by the
same unobserved characteristics and the decision is jointly made
\citep{Schultz2005}.
In addition, many of the outcomes of interest, such as children's
completed education, will not be known until many years later.

Because of the issues in identifying causal effects, the Matlab
experiments described above contribute most of the credible research in
this area.
Despite the substantial reduction in fertility that followed from the
differential access to family planning, there is little evidence of
significant effects on the school enrollments of boys or girls
\citep{Sinha2005}.
Using a different approach, there is some evidence that younger boys
completed more schooling with access to the program, but the effect is
smaller for older children, and not statistically significant for girls
of any age \citep{Joshi2007}.
The effect of the program on labor force participation is positive for
both boys and girls, but only significant for boys.
Furthermore, the effect of the program on our most common measure of
health, the height of children, is unclear, with some research
suggesting no significant differences in height for children less than
15 years old across treatment and control areas \citep{Joshi2007} and
other finding a significant effect \citep{Barham2012}.

Even though the results on education and health are mixed, children in
the treatment areas are significantly more likely to survive
\citep{Joshi2007}.
Having access to the program reduced under-five child mortality by five
percentage points.
In addition, preventive health inputs were used more frequently in the
treatment areas, as are prenatal care and tetanus inoculations for
mothers.
The substantial decline in child mortality point to substantial
improvements in early child health as a result of the program.

For adults, there is substantial evidence that women benefit from access
to the program \citep{Joshi2007}.
Women in the treatment areas had substantially higher BMI, which is
correlated with better health in a malnourished population like the one
in Matlab.
They were also more likely to have access to drinking and
cleaning/bathing water sources within the family compound.
Furthermore, treated women with more education lived in higher-valued
homesteads, agriculture, and owned more nonagricultural or financial
assets, and earned larger market incomes.

One argument for why we sometimes fail to find substantial changes in
fertility from access to family planning is that many people in
developing countries had little incentive to reduce the number of
children; the opportunity cost of women's time is low and children are
potentially productive on the family farm or can serve as old age
security \citep{Banerjee2014,Lambert2016}.
As a result, rather than focusing on the supply of family planning, some
economists emphasize policies that influence fertility demand, such as
household poverty and girls' schooling
\citep{pritchett94a,DasGupta2011}.

The most important of these policies is women's schooling
\citep{schultz02}.
The basic idea is that children require both parents' time and goods and
these inputs combine to produce a child and its traits.
The most important input is the mother's time.
Not only does pregnancy take its toll on the mother's productivity in
the labor market, children also require a substantial amount of time
after they are born and until they are able to fend for themselves.
With increasing education come increases in wages and productivity.
Hence, as women receive more education their (potential) wage increases,
which means that the opportunity cost of their time also increases.
In other words, the more education the mother has the higher is the cost
of having children in terms of foregone income.
Furthermore, even the perception of an increased opportunity cost lead
to a postponement of marriage and fertility and lower desired fertility
\citep{Jensen2012}.

The increase in the opportunity cost of children with more education is
not the only potential explanation for why education is associated with
lower fertility.
Higher education of women is also associated with significantly better
child health, although it is less clear what exactly it is about
education that leads to better child health
\citep{Thomas1991,Glewwe1999,Kovsted2002}.
The better health outcomes allow women to achieve their preferred number
with fewer births.
In addition, more education may lead to a better bargaining position for
women and if women prefer to have fewer children than men this would
reduce fertility.%
\footnote{
\citet{Ainsworth1996} reviews other potential explanations.}

\section{Conclusion}

\bibliographystyle{aer}
\bibliography{collection}

\end{document}
